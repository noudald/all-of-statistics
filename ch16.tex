\section*{Chapter 16 - Causal Inference}

I didn't really understand this chapter.
I'm not 100\% confident the solutions are correct.

\subsection*{Solution 16.1}

Create the following population.

\begin{table}[H]
    \centering
    \begin{tabular}{l|l||l|l}
    X & Y & $C_0$ & $C_1$ \\ \hline \hline
    0 & 1 & 1    & 0    \\ \hline
    0 & 0 & 0    & 0    \\ \hline
    1 & 1 & 1    & 1
    \end{tabular}
\end{table}
With this population we have $\theta = E(C_1) - E(C_0) = \frac{1}{3} - \frac{2}{3} = -\frac{1}{3} < 0$ and $\alpha = E(Y|X=1) - E(Y|X=0) = 1 - \frac{1}{2} = \frac{1}{2} > 0$.


\subsection*{Solution 16.2}

Let $X$ be randomly assigned.
We have
\begin{equation*}
    r(x) = E(Y|X = x)
        = \int_{X} C_{z}(x) f(z|x) dz
        = \int_{X} C_{z}(x) f(z) dz
        = E(C(X)).
\end{equation*}
As a counter example, let $C(x) = 0$ if $x < 0$ and $C(x) = 1$ otherwise on $x \in [-1, 1]$.
Assign $X$ such that $P(X < 0) = 1$ and $P(X \geq 0) = 0$.
Then $E(C(X)) = \frac{1}{2}$, but $r(x) = E(Y|X = x) = 0$, so $\theta(x) \neq r(x)$.


\subsection*{Solution 16.3}

Let $(X_1, Y_1), (X_2, Y_2), ..., (X_n, Y_n)$ be binary measurements from an observational study.
Following the hint,
\begin{equation*}
    \begin{split}
        \theta
            &= E(C_1) - E(C_0) \\
            &= E(C_1|X=1)P(X=1) + E(C_1|X=0)P(X=0) - (E(C_0|X=1)P(X=1) + E(C_0|X=0)P(X=0)) \\
            &= (E(C_1|X_1)P(X=1) - E(C_0|X=1)P(X=1)) + (E(C_1|X=0)P(X=0) - E(C_0|X=0)P(X=0)).
    \end{split}
\end{equation*}
Note that
\begin{equation*}
    -P(X=0) \leq (E(C_1|X=0)P(X=0) - E(C_0|X=0)P(X=0)) \leq P(X=1).
\end{equation*}
Hence we can estimate $L \leq \theta \leq U$, where
\begin{equation*}
    \begin{split}
        L &= \frac{1}{n_1} \sum X_i (Y_i - 1) - \frac{1}{n_0} \sum (X_i - 1)Y_i - P(X=0), \\
        U &= \frac{1}{n_1} \sum X_i (Y_i - 1) - \frac{1}{n_0} \sum (X_i - 1)Y_i + P(X=1),
    \end{split}
\end{equation*}
where $n_0$ is the number of samples with $X_i = 0$ and $n_1 = n - n_0$.
Furthermore, note that $U - L = P(X=0) + P(X=1) = 1$.
So this estimation is bounded by $1$.


\subsection*{Solution 16.4}

I don't get this exercise.
I think the idea behind the exercise is to create something similar as Firgure 16.2, but for more variables.
But I fell like I'm missing some of the tools or definitions to do so.


\subsection*{Solution 16.5}

Again, I don't understand well what is being asked.
This is my take:
Note that $m_0 = \lim_{y \to \infty} F^{-1}(\frac{1}{2}, y)$ and $m_1 = \lim_{x \to \infty} F^{-1}(x, \frac{1}{2})$.
So, $\theta = m_1 - m_0 = \lim_{x \to \infty} \lim_{y \to \infty} \left(F^{-1}(x, \frac{1}{2}) - F^{-1}(\frac{1}{2}, y)\right)$.

\section*{Chapter 16 - Causal Inference}

\subsection*{Solution 16.1}

Create the following population.

\begin{table}[H]
\begin{tabular}{l|l||l|l}
X & Y & $C_0$ & $C_1$ \\ \hline \hline
0 & 1 & 1    & 0    \\ \hline
0 & 0 & 0    & 0    \\ \hline
1 & 1 & 1    & 1
\end{tabular}
\end{table}
With this population we have $\theta = E(C_1) - E(C_0) = \frac{1}{3} - \frac{2}{3} = -\frac{1}{3} < 0$ and $\alpha = E(Y|X=1) - E(Y|X=0) = 1 - \frac{1}{2} = \frac{1}{2} > 0$.


\subsection*{Solution 16.2}

Let $X$ be randomly assigned.
We have
\begin{equation*}
    r(x) = E(Y|X = x)
        = \int_{X} C_{z}(x) f(z|x) dz
        = \int_{X} C_{z}(x) f(z) dz
        = E(C(X)).
\end{equation*}
As a counter example, let $C(x) = 0$ if $x < 0$ and $C(x) = 1$ otherwise on $x \in [-1, 1]$.
Assign $X$ such that $P(X < 0) = 1$ and $P(X \geq 0) = 0$.
Then $E(C(X)) = \frac{1}{2}$, but $r(x) = E(Y|X = x) = 0$, so $\theta(x) \neq r(x)$.

\section*{Chapter 19 - The log-linear model}

\subsection*{Solution 19.1}

\begin{align*}
    p_{00} &= e^{\beta_1}, & p_{10} &= e^{\beta_1 + \beta_2}, \\
    p_{01} &= e^{\beta_1 + \beta_3}, & p_{02} &= e^{\beta_2 + \beta_4}, \\
    p_{11} &= e^{\beta_1 + \beta_2 + \beta_3 + \beta_5}, & p_{12} &= e^{\beta_1 + \beta_2 + \beta_4 + \beta_6}.
\end{align*}


\subsection*{Solution 19.2}

\begin{itemize}
    \item[$\rightarrow$)] If $X_b \bot X_c | X_a$, then
        \begin{equation*}
            \frac{f(x_a, x_b, x_c)}{f(x_a)}
                = f(x_b, x_c| x_a) = f(x_b|x_a) f(x_c|x_a)
                = \frac{f(x_a, x_b)}{f(x_a)} \frac{f(x_a, x_c)}{f(x_a)}.
        \end{equation*}
        Rearranging the equation we find
        \begin{equation*}
            f(x_a, x_b, x_c) = f(x_a, x_b) \frac{f(x_a, x_c)}{f(x_a)}
                := g(x_a, x_b) h(x_a, x_c).
        \end{equation*}
    \item[$\leftarrow$)] Note that
        \begin{equation*}
            f(x_b, x_c|x_a) = \frac{f(x_a, x_b, x_c)}{f(x_a)}
                = \frac{g(x_a, x_b) h(x_a, x_c)}{f(x_a)}
                = g'(x_b) h'(x_c).
        \end{equation*}
        Using solution 2.12 we have $X_b \bot X_c | X_a$.
\end{itemize}


\subsection*{Solution 19.3}

Let $\mathcal{G} = (E, V)$ be a graphical model.
Take $\log(f(x)) = \sum_{A \subset S} \psi_A(x)$ with $\psi_A(x) = 0$ iff $A$ is non-empty and for all $\{i, j\} \in A$ and $(i, j) \notin E$.
Let $A \subset S$ such that $\psi_A = 0$ and $A \subset B \subset S$.
Let $\{i, j\} \in A$, then $(i, j) \notin E$.
Since $A \subset B$, $\{i, j\} \in B$, and hence $\psi_B = 0$.
Therefore, $\mathcal{G}$ is hierarchical.

The opposite is not true as you can see in example 19.11.


\subsection*{Solution 19.4}

\begin{itemize}
    \item[(a)]
        \tikz \graph {
            "$X_1$" -- {"$X_2$", "$X_3$"} -- "$X_4$"
        };
    \item[(b)] $X_1 \bot X_4 | \{X_2, X_3\}$, $X_2 \bot X_3 | \{X_1, X_4\}$.
    \item[(c)] The graph is not graphical, because $(1, 2) \in E$, but $\psi_{123} = 0$.
        The graph is hierarchical as $\emptyset$ is the only non-zero subset.
\end{itemize}


\subsection*{Solution 19.5}

In this particular case it's easier to calculate ${}^2\log(f)$.
Note that $\log(f) = \frac{{}^2\log(f)}{\log(2)}$.
So we can recover $\log(f)$ from ${}^2\log(f)$.
We have
\begin{equation*}
    {}^2\log(f)(x_1, x_2, x_3) = 1 + 3x_1 x_2 + 2x_3 + x_1x_3,
\end{equation*}
and
\begin{equation*}
    \log(f) = \psi_{\emptyset} + \psi_1 + \psi_2 + \psi_3 + \psi_{13}.
\end{equation*}
The graph is given by
\begin{center}
\tikz \graph {
    "$X_1$" -- {"$X_2$", "$X_3$"} -- "$X_4$"
};
\end{center}
In particular, $X_1 \bot X_2$ and $X_2 \bot X_3$.

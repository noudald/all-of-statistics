\section*{Chapter 8 - The Bootstrap}

\subsection*{Solution 8.1}

See code.


\subsection*{Solution 8.2}

See code.


\subsection*{Solution 8.3}

See code.


\subsection*{Solution 8.4}

Regard $X_1, X_2, ..., X_n$ as $n$ different buckets.
A bootstrap sample $X_1*, X_2^*, ..., X_n^*$ can be seen as a non-negative integer valued vector $(y_1, y_2, ..., y_n)$ where $y_i$ represents the number of times $X_i$ is chosen.
Or, in other words, $y_i$ is the number of times a ball is put in bucket $X_i$.
We look for all non-negative valued vector solutions $(y_1, y_2, ..., y_n)$ such that $y_1 + y_2 + ... + y_n = n$ and $y_i \geq 0$.
Or, equivalent, all $(\tilde{y}_1, \tilde{y}_2, ..., \tilde{y}_n)$ such that $\tilde{y}_1 + \tilde{y}_2 + ... + \tilde{y}_n = 2n$ and $\tilde{y}_i = y_1 + 1 \geq 1$.
This is the stars and bars problem.
Consider $2n$ stars $\star \star \ldots \star$ which we want to partition into $n$ non-empty sets.
There are $2n - 1$ places between the stars where we can put $n$ bars to split the stars.
Therefore, there are $\binom{2n - 1}{n}$ possible $n$ non-empty partitions of $2n$ stars.
Equivalent, there are $\binom{2n - 1}{n}$ unique vectors $(y_1, y_2, ..., y_n)$ such that $y_1 + y_2 + ... + y_n = n$ with $y_n \geq 0$.
Which gives the number of unique bootstrap draws of $Y_1, Y_2, ..., Y_n$ with replacement.


\subsection*{Solution 8.5}

Let $X_1, X_2, ..., X_n$ be i.i.d. random variables with $E(X_i) = \mu$ and $V(X_i) = \sigma^2$.
Let $X_1^*, X_2^*, ..., X_n^*$ be the bootstrap sample.
Define $\overline{X}_n^* = \frac{1}{n} \sum_{i=1}^n X_i^*$.
Note that
\begin{equation*}
    E(X^*_i | X_1, X_2, ..., X_n)
        = \sum_{i = 1}^n \frac{1}{n} X_i
        = \overline{X}_n.
\end{equation*}
Therefore,
\begin{equation*}
    E(\overline{X}_n^* | X_1, X_2, ..., X_n)
        = \frac{1}{n} \sum_{i = 1}^n E(X_i^* | X_1, X_2, ..., X_n)
        = \frac{1}{n} \sum_{i = 1}^n \overline{X}_n
        = \overline{X}_n.
\end{equation*}
By the rule of iterated expectations, Theorem 3.24,
\begin{equation*}
    E(\overline{X}_n^*)
        = E(E(\overline{X}_n^* | X_1, X_2, ..., X_n))
        = E(\overline{X}_n)
        = \mu.
\end{equation*}
Next we have
\begin{equation*}
    V(\overline{X}_n^*|X_1, X_2, ..., X_n)
        = \frac{1}{n^2} \sum_{i = 1}^n V(X_i^*|X_1, X_2, ..., X_n)
        = \frac{1}{n^2} \sum_{i = 1}^n \sum_{j = 1}^n \frac{1}{n} (X_j - E(X_j))^2
        = \frac{1}{n^2} \sum_{i = 1}^n (X_i - \overline{X}_n)^2.
\end{equation*}
By Theorem 3.27 we have
\begin{equation*}
    \begin{split}
        V(\overline{X}_n^*)
            &= E(V(\overline{X}_n^*|X_1, X_2, ..., X_n)) + V(E(\overline{X}_n^*|X_1, X_2, ..., X_n)) \\
            &= E\left(\frac{1}{n^2} \sum_{i = 1}^n (X_i - \overline{X}_n)^2\right) + V(\overline{X}_n) \\
            &= \frac{n - 1}{n^2} \sigma^2 + \frac{\sigma^2}{n}
            = \frac{2n - 1}{n^2} \sigma^2,
    \end{split}
\end{equation*}
where we used Theorem 3.17 to calculate
\begin{equation*}
    E\left(\frac{1}{n^2} \sum_{i = 1}^n (X_i - \overline{X}_n)^2\right)
        = \frac{n - 1}{n^2} E\left( \frac{1}{n - 1} \sum_{i = 1}^n (X_i - \overline{X}_n)^2 \right)
        = \frac{n - 1}{n^2} \sigma^2.
\end{equation*}

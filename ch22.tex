\section*{Chapter 22 - Classification}

\subsection*{Solution 22.1}

Suppose there is an $h$ such that $L(h) < L(h^*)$, then
\begin{equation*}
    \int_0^1 P(h(x) = Y | X = x) P(x) dx > \int_0^1 P(h^*(x) = Y | X = x) P(x) dx.
\end{equation*}
With sufficient properties on the distribution there is an $x$ such that $P(h(x) = Y | X = x) > P(h^*(x) = Y | X = x)$.
But this contradicts the choice of $h^*$.


\subsection*{Solution 22.2}

In general we have
\begin{equation*}
    P(y|x) = \frac{P(x|y)P(y)}{P(x)}
        = \frac{P(x|y)f(y)}{P(x|0)f(0) + P(x|1)f(1)}
        = \frac{P(x|y) \pi_y}{P(x|0) \pi_0 + P(x|1) \pi_1},
\end{equation*}
where $\pi_y = P(Y = y)$.
In particular
\begin{equation*}
    P(1|x) = \frac{P(x|1) \pi_1}{P(x|0) \pi_0 + P(x|1) \pi_1}
        = \frac{P(x|1)}{P(x|0) \frac{\pi_0}{\pi_1} + P(x|1)}
        > \frac{1}{2},
\end{equation*}
if and only if
\begin{equation*}
    \frac{\pi_1}{\pi_0} f(x|1) > f(x|0).
\end{equation*}
Expanding $f(x|0)$ and $f(x|1)$, and taking the log on both sides gives
\begin{equation*}
    r_1^2 < r_0^2 + \log\left(\frac{|\Sigma_0|}{|\Sigma_1|}\right) + 2\log\left(\frac{\pi_1}{\pi_0}\right).
\end{equation*}


\subsection*{Solution 22.3}

See code.


\subsection*{Solution 22.4}

I have no idea how to solve this.


\subsection*{Solution 22.5}

See code.


\subsection*{Solution 22.6}

Similar to 22.4. I don't understand VC-theory.


\subsection*{Solution 22.7}

Suppose there is a linear classifier $r(x) = \beta_1 x + \beta_0$ that perfectly classifies the data.
Note that $Y_i = 1$ if $r(X_i) > \frac{1}{2}$.
So we have $Y_i = 1$ if and only if $X_i > \frac{1}{2 \beta_1} (1 - \beta_0)$.
But in that case there will always be some $X_i$ with $Y_i = 0$, but $X_i > \frac{1}{2 \beta_1} (1 - \beta_0)$.

Note that the kernelized data $Z_i = (X_i, X_i^2)$ can be linearly seperated.
Indeed, we can take $r(Z_i) = \frac{1}{2} X_i^2$.


\subsection*{Solution 22.8}

See code.


\subsection*{Solution 22.9}

See code.


\subsection*{Solution 22.10}

The CDF is given by
\begin{equation*}
    F(t) = P(R < t)
        = 1 - P(R \geq t)
        = 1 - \prod_{i = 1}^n P(X_i \geq t)
        = 1 - \prod_{i = 1}^n (1 - v_d(t))
        = 1 - (1 - t^2 v_d(1))^n.
\end{equation*}
Hence, we have
\begin{equation*}
    F^{-1}(q) = \left(\frac{1 - (1 - q)^{\frac{1}{n}}}{v_d(1)}\right)^{\frac{1}{d}}.
\end{equation*}
The median is given by
\begin{equation*}
    F^{-1}\left(\frac{1}{2}\right) = \left(\frac{1 - (\frac{1}{2})^{\frac{1}{n}}}{v_d(1)}\right)^{\frac{1}{d}}.
\end{equation*}
See code for last part of the exercise.
If $n = 100$, $d \geq 6$; $n = 1000$, $d \geq 8$; $n = 10000$, $d \geq 9$.


\subsection*{Solution 22.11}

See code.


\subsection*{Solution 22.12}

See code.


\subsection*{Solution 22.13}

I have no idea.

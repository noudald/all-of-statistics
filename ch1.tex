\section*{Chapter 1 - Probability of Finite Sample Spaces}

\subsection*{Solution 1}

Let $A_1 \subset A_2 \subset A_3 \subset ...$.
Define $B_i = A_i \backslash \cup_{j < i} A_j$.

(a) Let $a \in A_n$.
Let $i \leq n$ the smallest $i$ such that $a \in A_n$.
Then $a \in A_i \backslash \cup_{j < i} A_j = B_i$.
Let $b \in B_i$ for some $i$.
Then $b \in A_i \subset A_n$.
Therefore $A_n = \cup_{i = 1}^n B_i$.

(b) Take $a \in \cup_{i = 1}^{\infty} A_i$.
There is an $n$ such $a \in A_n = \cup_{i = 1}^n B_i$.
So $a \in \cup_{i = 1}^{\infty} B_i$.
Let $b \in \cup_{i = 1}^{\infty} B_i$, then $b \in B_n$ for some $n$.
Therfore $b \in A_n$, hence $b \in \cup_{i=1}^{\infty} A_n$.
So $\cup_{i=1}^{\infty} A_i = \cup_{i=1}^n B_i$.


\subsection*{Solution 2}

\begin{itemize}
    \item[(i)] Let $A_1 = \emptyset$, $A_2 = \Omega$, then $A_1 \cap A_2 = \emptyset$ and $1 = P(\Omega) = P(A_1 \cup A_2) = P(A_1) + P(A_2) = P(\emptyset) + P(\Omega) = P(\emptyset) + 1$.
        Hence $P(\emptyset) = 1 - 1 = 0$.
    \item[(ii)] $A \subset B$, let $A_1 = A$, $A_2 = B \backslash A$. Then $A_1 \cap A_2 = \emptyset$ and $P(A) + P(B \backslash A) = P(A_1) + P(A_2) = P(A_1 \cup A_2) = P(B)$.
        Because $P(X) \geq 0$ for all $X$, $P(A) = P(B) - P(B \backslash A) \leq P(B)$.
    \item[(iii)] $A \subset \Omega$, so from (ii) $0 \leq P(A) \leq P(\Omega) = 1$.
    \item[(iv)] $1 = P(\Omega) = P(A \cup \Omega \backslash A) = P(A) + P(\Omega \backslash A) = P(A) + P(A^c)$.
        So $P(A^c) = 1 - P(A)$.
    \item[(v)] Take $A_1 = A$, $A_2 = B$, $A_i = \emptyset$ for $i > 2$.
\end{itemize}


\subsection*{Solution 3}

\begin{itemize}
    \item[(a)] Let $b \in B_{n+1}$, then $b \in A_m$ for all $m \geq n + 1 > n$, so $b \in B_n$.
        Hence $B_1 \supset B_2 \supset ...$.
        Let $c \in C_n$, then $c \in A_m$ for all $m \geq n$.
        So $c \in A_p$ for all $p \geq n + 1 > n$, so $c \in C_{n+1}$.
        Hence $C_1 \subset C_2 \subset ...$.
    \item[(b)] $\rightarrow$, let $\omega \in \cap B_n$, then $\omega \in B_n$ for all $n$, hence there is an $m$ such that for all $i \geq m$, $\omega \in A_i$.
        $\leftarrow$, let $\omega$ be in infinite $A_i$, but $\omega \notin \cap B_n$.
        Then there is a $B_m$ such that $\omega \notin B_m = \cup_{i \geq m} A_m$.
        But then $\omega \notin A_i$ for all $i \geq m$, which contradics that there are infinite $A_i$ containing $\omega$.
    \item[(c)] $\rightarrow$, let $\omega \in \cup C_i$, then there is an $m$ such that $\omega \in C_m = \cap_{i \geq m} A_i$.
        So $\omega$ is contained in $A_m, A_{m+1}, ...$.
        $\leftarrow$, let $m > 0$ such that $\omega$ in $A_m, A_{m+1}, ...$.
        We can do this because there are only a finite number of $A_i$ not containing $\omega$.
        Then $\omega \in \cap_{i \geq m} A_i = C_m$, so $\omega \in \cup C_i$.

\end{itemize}


\subsection*{Solution 4}

\begin{itemize}
    \item[(a)] $x \in (\cup A_i)^c$, iff $x \notin \cup A_i$, iff $x \notin A_i$ for all $i$, iff $x \in A_i^c$ for all $i$, iff $x \in \cap A_i^c$.
    \item[(b)] $x \in (\cap A_i)^c$, iff $x \notin \cap A_i$, iff $x \notin A_i$ for some $i$, iff $x \in A_i^c$ for some $i$, iff $x \in \sum A_i^c$.
\end{itemize}


\subsection*{Solution 5}

The sample space is $S = \{ x_1 x_2 ... x_k : \exists i > n, x_i = H, x_n = H, \forall j \neq i, j \neq n, x_j = T\}$.
The probability that $k$ tosses are required is
\begin{equation*}
    \sum_{j < k - 1} (1 - p)^j p (1 - p)^{k - j} p
        = (k - 1) p^2 (1 - p)^{k - 2}
        = \frac{k - 1}{2^k}.
\end{equation*}


\subsection*{Solution 6}

Assume there exists a uniform probability $P$ on the discrete infinite sample space $\Omega$.
Because $\sum_{x \in \Omega} P(x) = 1$, there is a $y \in \Omega$ such that $P(y) = c > 0$.
As $|\{y\}| = 1$, $1 = \sum_{x \in \Omega} P(x) = \sum_{x \in \Omega} P(y) = \sum_{x \in \Omega} c = c |\Omega|$.
So $\Omega$ is a finite set.
But this contradicts with the assumption that $\Omega$ is an infinite set.
So there doesn't exist a uniform probability $P$ on a discrete infinite sample space.


\subsection*{Solution 7}

Define $B_n = A_n \backslash \cup_{i = 1}^n A_i$.
Then $B_i \cap B_j = \emptyset$ when $i \neq j$, and $\cup B_n = \cup A_n$.
So $P(\cup A_n) = P(\cup B_n) = \sum P(B_n)$.
As $B_n \subset A_n$, $P(B_n) \leq P(A_n)$.
Therefore we have $P(\cup A_n) = \sum P(B_n) \leq \sum P(A_n)$.


\subsection*{Solution 8}

Proving $P(\cap A_i) = 1$ is equivalent to proving $P((\cap A_i)^c) = P(\cup A_i^c) = 0$.
Take disjoint sets $B_n = A_n^c \backslash \cup_{i < n} A_i^c$.
Note that $B_n^c \subset A_n^c$, so $P(B_n^c) \leq P(A_n^c) = 1 - P(A_n) = 0$.
So
\begin{equation*}
    P(\cup_{i = 1}^n A_i^c)
        = P(\cup_{i = 1}^n B_i)
        = \sum_{i = 1}^n P(B_i)
        = 0.
\end{equation*}


\subsection*{Solution 9}

\begin{itemize}
    \item[1.] $P(X, B) \geq 0$ and $P(B) \geq 0$, so $P(X|B) = P(X, B) / P(B) \geq 0$.
    \item[2.] $P(\Omega|B) = P(\Omega \cap B) / P(B) = P(B) / P(B) = 1$.
    \item[3.] Let $A_1, A_2, ...$ be disjoint, then $A_1 \cap B, A_2 \cap B, ...$ are disjoint, and
        \begin{equation*}
            P(\cup_{i = 1}^{\infty} A_i | B)
                = \frac{P(\cup_{i = 1}^{\infty} A_i \cap B)}{P(B)}
                = \sum_{i = 1}^{\infty} \frac{P(A_i \cap B)}{P(B)}
                = \sum_{i = 1}^{\infty} P(A_i | B).
        \end{equation*}
\end{itemize}


\subsection*{Solution 10}

When we always pick door 1, the sample space is $\Omega = \{(1, 2), (1, 3), (2, 3), (3, 2)\}$, with probabilities $P(1, 2) = 1/6$, $P(1, 3) = 1/6$, $P(2, 3) = 1/3$, and $P(3, 2) = 1/3$.
We have
\begin{equation*}
    \begin{split}
        P(\omega_1 = 1 | \omega_2 = 2) &= \frac{P(\omega_1 = 1, \omega_2 = 2)}{P(\omega_2 = 2)} = \frac{1}{3}, \\
        P(\omega_1 = 3 | \omega_2 = 2) &= \frac{P(\omega_1 = 2, \omega_2 = 2)}{P(\omega_2 = 2)} = \frac{2}{3}, \\
        P(\omega_1 = 1 | \omega_2 = 3) &= \frac{P(\omega_1 = 1, \omega_2 = 2)}{P(\omega_2 = 3)} = \frac{1}{3}, \\
        P(\omega_2 = 3 | \omega_2 = 3) &= \frac{P(\omega_1 = 3, \omega_2 = 2)}{P(\omega_2 = 3)} = \frac{2}{3}. \\
    \end{split}
\end{equation*}
We conclude that it's better to switch to doors.


\subsection*{Solution 11}

Suppose $A \bot B$, i.e. $P(AB) = P(A)P(B)$, then
\begin{equation*}
    P(A^c B^c) = P((A \cup B)^c)
        = 1 - P(A \cup B)
        = 1 - P(A) - P(B) + P(A)P(B)
        = (1 - P(A))(1 - P(B))
        = P(A^c) P(B^c).
\end{equation*}


\subsection*{Solution 12}

I think this question is not well-defined and similar to the boy-girl paradox.
We could intrepid the questions in multiple ways and get both right answers $\frac{1}{2}$ and $\frac{1}{3}$.


\subsection*{Solution 13}

\begin{itemize}
    \item[(a)] $\Omega = \{ HH^kT, TT^kH: k \geq 0 \}$.
    \item[(b)] $P(HHT, TTH) = 2 \frac{1}{2^3} = \frac{1}{4}$.
\end{itemize}

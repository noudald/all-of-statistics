\section*{Chapter 5 - Convergence of Random Variables}

\subsection*{Solution 1}

Let $X_1, X_2, ..., X_n$ i.i.d. with $E(X_i) = \mu$ and $V(X_i) = \sigma^2$.

\begin{itemize}
    \item[(a)] See solution 3.8.
    \item[(b)] We write
        \begin{equation*}
            \begin{split}
                S_n^2 &= \frac{1}{n - 1} \sum_{i = 1}^n (X_i - \overline{X}_n)^2 \\
                    &= \frac{1}{n - 1} \sum_{i = 1}^n (X_i^2 - 2 \overline{X}_n X_i + \overline{X}_n^2) \\
                    &= \frac{n}{n - 1} \frac{1}{n} \sum_{i = 1}^n X_i^2 - \frac{n - 2}{n - 1} \overline{X}_n^2
                    = c_n \frac{1}{n} \sum_{i = 1}^n X_i^2 - d_n \overline{X}_n^2,
            \end{split}
        \end{equation*}
        where $c_n, d_n \to 1$ as $n \to \infty$.
        By the weak law of large numbers $\frac{1}{n} \sum_{i = 1} X_i^2 \xrightarrow{P} E(X_i^2) = \mu^2 + \sigma^2$ and $\overline{X}_n^2 \xrightarrow{P} \mu^2$.
        Using Theorem 5.5, $S_n^2 \xrightarrow{P} \sigma^2$.
\end{itemize}


\subsection*{Solution 2}

Let $X_1, X_2, ...$ be a sequence of random variables.
\begin{itemize}
    \item[$\rightarrow$)] Suppose $X_n \xrightarrow{qm} b$, i.e. $E((X_n - b)^2) \to 0$ as $n \to \infty$.
        We calculate
        \begin{equation}
            \label{eqn:5.2.1}
            E((X_n - b)^2) = E(X_n^2) - 2bE(X_n) + b^2
                = V(X_n) + (E(X_n) - b)^2
                \to 0,
        \end{equation}
        as $n \to \infty$.
        Both $V(X_n)$ and $(E(X_n) - b)^2$ are non-negative, so we must have $V(X_n) \to 0$ and $(E(X_n) - b)^2 \to 0$ as $n \to \infty$.
        Finally, if $(E(X_n) - b)^2 \to 0$, then $E(X_n) \to b$, as $n \to \infty$.
    \item[$\leftarrow$)] Suppose $E(X_n) \to b$ and $V(X_n) \to 0$ as $n \to \infty$.
        Using (\ref{eqn:5.2.1}), $E((X_n - b)^2) = V(X_n) + (E(X_n) - b)^2 \to 0$ as $n \to \infty$.
\end{itemize}


\subsection*{Solution 3}

Let $X_1, X_2, ..., X_n$ be i.i.d. random variables.
Let $\mu = E(X)$ and $\sigma^2 = V(X)$ be finite.
Take the sample mean $\overline{X}_n = \frac{1}{n} \sum X_i$.
We have
\begin{equation*}
    E((\overline{X}_n - \mu)^2) = V(\overline{X}_n)
        = \frac{1}{n^2} \sum_{i = 1}^n V(X_i)
        = \frac{\sigma^2}{n} \to 0,
\end{equation*}
when $n \to \infty$.
Therefore $X_n \xrightarrow{qm} \mu$ as $n \to \infty$.


\subsection*{Solution 4}

Let $X_1, X_2, ...$ be i.i.d. random variables defined by $P(X_n = \frac{1}{n}) = 1 - \frac{1}{n^2}$ and $P(X_n = n) = \frac{1}{n^2}$.

\begin{itemize}
    \item[(a)] Note that $E(X_n) = \frac{1}{n}(1 - \frac{1}{n^2}) + n \frac{1}{n^2} = 2 \frac{1}{n} - \frac{1}{n^3} \to 0$ as $n \to \infty$.
        But $V(X_n) = E(X_n^2) - E(X_n)^2 = E(X_n^2) = \frac{1}{n^2}(1 - \frac{1}{n^2}) + n^2 \frac{1}{n^2} = 1 + \frac{1}{n^2} - \frac{1}{n^4} \to 1$ as $n \to \infty$.
        Therefore, by exercise 5.2, $X_n$ cannot converge in quadratic mean.
    \item[(b)] Let $\epsilon > 0$, choose $n$ large enough such that $\frac{1}{n} < \epsilon$.
        We have $P(|X_n| > \epsilon) = \frac{1}{n^2} \to 0$ when $n \to \infty$.
        Therefore $X_n \xrightarrow{P} 0$.
\end{itemize}


\subsection*{Solution 5}

Let $X_1, X_2, ... \sim \mathrm{Bernoulli}(p)$ i.i.d.
Note
\begin{equation*}
    E(\frac{1}{n} \sum_{i = 1}^n X_i^2) = \frac{1}{n} \sum_{i = 1}^n E(X_i^2)
        = \frac{1}{n} \sum_{i = 1}^n p
        = p,
\end{equation*}
and
\begin{equation*}
    V(\frac{1}{n} \sum_{i = 1}^n X_i^2) = \frac{1}{n^2} \sum_{i = 1}^n V(X_i^2)
        = \frac{1}{n^2} \sum_{i = 1}^n (E(X_i^2) - E(X_i)^2)
        = \frac{1}{n^2} \sum_{i = 1}^n p(p - 1)
        = \frac{p(p - 1)}{n}.
\end{equation*}
So $E(\frac{1}{n} \sum X_i^2) \to p$ and $V(\frac{1}{n} \sum X_i^2) \to 0$ as $n \to \infty$.
By Exercise 5.2 we have $\frac{1}{n} \sum X_i^2 \xrightarrow{qm} p$.
By Theorem 5.4 we have $\frac{1}{n} \sum X_i^2 \xrightarrow{P} p$.


\subsection*{Solution 6}

Let $X_1, X_2, ..., X_{100}$ be i.i.d. random samples with $E(X_i) = 68$ and $V(X_i) = 26^2$.
By the Central Limit Theorem
\begin{equation*}
    Z_{100} = \sqrt{100}\, \frac{\overline{X}_{100} - 68}{26} \approx \mathrm{Normal}(0, 1).
\end{equation*}
Therefore
\begin{equation*}
    P(\overline{X}_{100} > 68) = P\left(\frac{26}{\sqrt{100}} Z_{100} + 26 > 26\right)
        = P(Z_{100} > 0)
        \approx 0.5.
\end{equation*}


\subsection*{Solution 7}

Let $\lambda_n = \frac{1}{n}$ and $X_n \sim \mathrm{Poisson}(\lambda_n)$ for $n = 1, 2, ...$.
\begin{itemize}
    \item[(a)] $P(|X_n| > \epsilon) = 1 - P(|X_n| \leq \epsilon) < 1 - P(X_n = 0) = 1 - e^{-\frac{1}{n}} \to 0$ as $n \to \infty$.
    \item[(b)] Almost the same, $P(|Y_n| > \epsilon) = P(|X_n| > \frac{\epsilon}{n}) = 1 - P(|X_n| \leq \frac{\epsilon}{n}) < 1 - P(X_n = ) \to 0 = 1 - e^{-\frac{1}{n}}$ when $n \to \infty$
\end{itemize}

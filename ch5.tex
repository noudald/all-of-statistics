\section*{Chapter 5 - Convergence of Random Variables}

\subsection*{Solution 1}

Let $X_1, X_2, ..., X_n$ i.i.d. with $E(X_i) = \mu$ and $V(X_i) = \sigma^2$.

\begin{itemize}
    \item[(a)] See solution 3.8.
    \item[(b)] We write
        \begin{equation*}
            \begin{split}
                S_n^2 &= \frac{1}{n - 1} \sum_{i = 1}^n (X_i - \overline{X}_n)^2 \\
                    &= \frac{1}{n - 1} \sum_{i = 1}^n (X_i^2 - 2 \overline{X}_n X_i + \overline{X}_n^2) \\
                    &= \frac{n}{n - 1} \frac{1}{n} \sum_{i = 1}^n X_i^2 - \frac{n - 2}{n - 1} \overline{X}_n^2
                    = c_n \frac{1}{n} \sum_{i = 1}^n X_i^2 - d_n \overline{X}_n^2,
            \end{split}
        \end{equation*}
        where $c_n, d_n \to 1$ as $n \to \infty$.
        By the weak law of large numbers $\frac{1}{n} \sum_{i = 1} X_i^2 \xrightarrow{P} E(X_i^2) = \mu^2 + \sigma^2$ and $\overline{X}_n^2 \xrightarrow{P} \mu^2$.
        Using Theorem 5.5, $S_n^2 \xrightarrow{P} \sigma^2$.
\end{itemize}


\subsection*{Solution 2}

Let $X_1, X_2, ...$ be a sequence of random variables.
\begin{itemize}
    \item[$\rightarrow$)] Suppose $X_n \xrightarrow{qm} b$, i.e. $E((X_n - b)^2) \to 0$ as $n \to \infty$.
        We calculate
        \begin{equation}
            \label{eqn:5.2.1}
            E((X_n - b)^2) = E(X_n^2) - 2bE(X_n) + b^2
                = V(X_n) + (E(X_n) - b)^2
                \to 0,
        \end{equation}
        as $n \to \infty$.
        Both $V(X_n)$ and $(E(X_n) - b)^2$ are non-negative, so we must have $V(X_n) \to 0$ and $(E(X_n) - b)^2 \to 0$ as $n \to \infty$.
        Finally, if $(E(X_n) - b)^2 \to 0$, then $E(X_n) \to b$, as $n \to \infty$.
    \item[$\leftarrow$)] Suppose $E(X_n) \to b$ and $V(X_n) \to 0$ as $n \to \infty$.
        Using (\ref{eqn:5.2.1}), $E((X_n - b)^2) = V(X_n) + (E(X_n) - b)^2 \to 0$ as $n \to \infty$.
\end{itemize}


\subsection*{Solution 3}

Let $X_1, X_2, ..., X_n$ be i.i.d. random variables.
Let $\mu = E(X)$ and $\sigma^2 = V(X)$ be finite.
Take the sample mean $\overline{X}_n = \frac{1}{n} \sum X_i$.
We have
\begin{equation*}
    E((\overline{X}_n - \mu)^2) = V(\overline{X}_n)
        = \frac{1}{n^2} \sum_{i = 1}^n V(X_i)
        = \frac{\sigma^2}{n} \to 0,
\end{equation*}
when $n \to \infty$.
Therefore $X_n \xrightarrow{qm} \mu$ as $n \to \infty$.


\subsection*{Solution 4}

Let $X_1, X_2, ...$ be i.i.d. random variables defined by $P(X_n = \frac{1}{n}) = 1 - \frac{1}{n^2}$ and $P(X_n = n) = \frac{1}{n^2}$.

\begin{itemize}
    \item[(a)] Note that $E(X_n) = \frac{1}{n}(1 - \frac{1}{n^2}) + n \frac{1}{n^2} = 2 \frac{1}{n} - \frac{1}{n^3} \to 0$ as $n \to \infty$.
        But $V(X_n) = E(X_n^2) - E(X_n)^2 = E(X_n^2) = \frac{1}{n^2}(1 - \frac{1}{n^2}) + n^2 \frac{1}{n^2} = 1 + \frac{1}{n^2} - \frac{1}{n^4} \to 1$ as $n \to \infty$.
        Therefore, by exercise 5.2, $X_n$ cannot converge in quadratic mean.
    \item[(b)] Let $\epsilon > 0$, choose $n$ large enough such that $\frac{1}{n} < \epsilon$.
        We have $P(|X_n| > \epsilon) = \frac{1}{n^2} \to 0$ when $n \to \infty$.
        Therefore $X_n \xrightarrow{P} 0$.
\end{itemize}


\subsection*{Solution 5}

Let $X_1, X_2, ... \sim \mathrm{Bernoulli}(p)$ i.i.d.
Note
\begin{equation*}
    E(\frac{1}{n} \sum_{i = 1}^n X_i^2) = \frac{1}{n} \sum_{i = 1}^n E(X_i^2)
        = \frac{1}{n} \sum_{i = 1}^n p
        = p,
\end{equation*}
and
\begin{equation*}
    V(\frac{1}{n} \sum_{i = 1}^n X_i^2) = \frac{1}{n^2} \sum_{i = 1}^n V(X_i^2)
        = \frac{1}{n^2} \sum_{i = 1}^n (E(X_i^2) - E(X_i)^2)
        = \frac{1}{n^2} \sum_{i = 1}^n p(p - 1)
        = \frac{p(p - 1)}{n}.
\end{equation*}
So $E(\frac{1}{n} \sum X_i^2) \to p$ and $V(\frac{1}{n} \sum X_i^2) \to 0$ as $n \to \infty$.
By Exercise 5.2 we have $\frac{1}{n} \sum X_i^2 \xrightarrow{qm} p$.
By Theorem 5.4 we have $\frac{1}{n} \sum X_i^2 \xrightarrow{P} p$.


\subsection*{Solution 6}

Let $X_1, X_2, ..., X_{100}$ be i.i.d. random samples with $E(X_i) = 68$ and $V(X_i) = 26^2$.
By the Central Limit Theorem
\begin{equation*}
    Z_{100} = \sqrt{100}\, \frac{\overline{X}_{100} - 68}{26} \approx \mathrm{Normal}(0, 1).
\end{equation*}
Therefore
\begin{equation*}
    P(\overline{X}_{100} > 68) = P\left(\frac{26}{\sqrt{100}} Z_{100} + 26 > 26\right)
        = P(Z_{100} > 0)
        \approx 0.5.
\end{equation*}


\subsection*{Solution 7}

Let $\lambda_n = \frac{1}{n}$ and $X_n \sim \mathrm{Poisson}(\lambda_n)$ for $n = 1, 2, ...$.
\begin{itemize}
    \item[(a)] $P(|X_n| > \epsilon) = 1 - P(|X_n| \leq \epsilon) < 1 - P(X_n = 0) = 1 - e^{-\frac{1}{n}} \to 0$ as $n \to \infty$.
    \item[(b)] Almost the same, $P(|Y_n| > \epsilon) = P(|X_n| > \frac{\epsilon}{n}) = 1 - P(|X_n| \leq \frac{\epsilon}{n}) < 1 - P(X_n = ) \to 0 = 1 - e^{-\frac{1}{n}}$ when $n \to \infty$
\end{itemize}


\subsection*{Solution 8}

Let $X_1, X_2, ..., X_{100}$ be i.i.d. random variables such that $X_i \sim \mathrm{Poisson}(1)$.
Let $Y = \sum X_i = n \overline{X}_n$.
By the Central Limit Theorem $\frac{\sqrt{n}}{\sigma} (\overline{X}_n - \mu) \approx Z \sim \mathrm{Normal}(0, 1)$.
We have
\begin{equation*}
    P(Y < 90) = P(\overline{X}_n < 0.9)
        \approx P(\frac{\sigma}{\sqrt{n}} Z + \mu < 0.9)
        = P(Z < \frac{\sqrt{n}}{\sigma} (0.9 - \mu))
        = \Phi(-1).
\end{equation*}


\subsection*{Solution 9}

Let $X$ be a discrete random variable such that $P(X = 1) = P(X = -1) = \frac{1}{2}$.
Define $X_n$ by $P(X_n = X) = 1 - \frac{1}{n}$ and $P(X_n = e^n) = \frac{1}{n}$.
\begin{itemize}
    \item[(a)] (Quadratic convergence)
        $E(X_n) = (1 - \frac{1}{n}) E(X) + \frac{1}{n} E(e^n) = \frac{1}{n}e^n \to \infty$ when $n \to \infty$.
        By exercise 5.2 $X_n$ doesn't converge quadratic to any distribution.
    \item[(b)] (Probability convergence)
        Let $\epsilon > 0$ and take $n$ such that $\frac{1}{n} < \epsilon$.
        Then $P(|X_n| > \epsilon) = 1$.
        So $P(X_n > \epsilon)$ does not converge to $0$ when $n \to \infty$.
        Therefore $X_n$ doesn't converge in probability.
    \item[(c)] (Distribution convergence)
        Let
        \begin{equation*}
            F_n(x) = \left\{ \begin{array}{lll}
                0 & \textrm{if} & x < -1, \\
                \frac{1}{2} - \frac{1}{2n} & \textrm{if} & -1 \leq x < 1, \\
                1 - \frac{1}{n} & \textrm{if} & 1 \leq x < e^n, \\
                1 & \textrm{if} & e^n < x,
            \end{array} \right.
            \quad
            F(x) = \left\{ \begin{array}{lll}
                0 & \textrm{if} & x < -1, \\
                \frac{1}{2} & \textrm{if} & -1 \leq x < 1, \\
                1 & \textrm{if} & 1 \leq x.
            \end{array} \right.
        \end{equation*}
        We have $X_n \sim F_n$.
        Let $X \sim \mathrm{Uniform}(-1, 1)$, then $X \sim F$.
        Note that $\lim_{n \to \infty} F_n(x) = F(x)$ for all $x$.
        Therefore $X_n \xrightarrow{D} X$.
\end{itemize}


\subsection*{Solution 10}

Let $Z \sim \mathrm{Normal}(0, 1)$.
Let $t > 0$ and $k \geq 1$ (I think this is wrong in the book, we cannot have $0 \leq k < 1$).
From Markov's inequality
\begin{equation*}
    P(|Z| > t) = P(|Z|^k > t^k)
        \leq \frac{E(|Z|^k)}{t^k}.
\end{equation*}
From Mill's inequality we have
\begin{equation*}
    P(|Z| > t) \leq \sqrt{\frac{2}{\pi}} \frac{e^{-t^2/2}}{t}.
\end{equation*}
Note that $\frac{E(|Z|)}{t} \leq \frac{E(|Z|^k)}{t^k}$.
So we only have to compare $E(|Z|)$ with $\sqrt{\frac{2}{\pi}} e^{-t^2/2}$.
We have
\begin{equation*}
    E(|Z|) = \int_{-\infty}^{\infty} |z| \frac{1}{\sqrt{2\pi}} e^{-z^2/2} dz
        = \frac{2}{\sqrt{2\pi}} \int_0^{\infty} z e^{-z^2/2} dz
        \geq \frac{2}{\sqrt{2\pi}} \int_t^{\infty} z e^{-z^2/2} dz
        = \sqrt{\frac{2}{\pi}} \left[ -e^{-z^2/2} \right]_{t}^{\infty}
        = \sqrt{\frac{2}{\pi}} e^{\frac{t^2}{2}}.
\end{equation*}
So Mill's inequality is always sharper than Markov's inequality.


\subsection*{Solution 11}

Let $X_1, X_2, ...$ be i.i.d. random variables defined by $X_n \sim \mathrm{Normal}(0, \frac{1}{n})$.
Let $X \sim F$ where the cumulative distribution function is defined by $F(x) = 0$ if $x < 0$ and $F(x) = 1$ if $x \geq 0$.
In other words $X = 0$.
Let $\epsilon > 0$, then by Mill's inequality
\begin{equation*}
    P(|X_n - X| > \epsilon) = P(|\sqrt{n} Z| > \epsilon)
        = P(|Z| > \frac{\epsilon}{\sqrt{n}})
        \leq \sqrt{\frac{2}{\pi}} \frac{1}{\epsilon} \frac{\sqrt{n}}{e^{-\frac{1}{2\epsilon} \sqrt{n}}}
        \to 0,
\end{equation*}
when $n \to \infty$.
So $X_n \xrightarrow{P} X$ as $n \to \infty$.
By Theorem 5.4 we have $X_n \xrightarrow{D} X$ as $n \to \infty$.


\subsection*{Solution 12}

Let $X$ and $X_1, X_2, ...$ be positive integer values random variables.
\begin{itemize}
    \item[($\rightarrow$)] Suppose $X_n \xrightarrow{D} X$, then $\lim_{n \to \infty} F_n(k) = F(k)$ for all $k \in \mathbb{N}$.
        Note that $P_n(X_n = k) = F_n(k) - F_n(k - 1)$.
        So we have
        \begin{equation*}
            \lim_{n \to \infty} P(X_n = k) = \lim_{n \to \infty} (F_n(k) - F_n(k - 1))
                = F(k) - F(k - 1)
                = P(X = k).
        \end{equation*}
    \item[($\leftarrow$)] Suppose that $\lim_{n \to \infty} P(X_n = k) = P(X = k)$ for all $k$.
        We have
        \begin{equation*}
            F(k) - F(k - 1) = P(X = k)
                = \lim_{n \to \infty} P(X_n = k)
                = \lim_{n \to \infty} (F_n(k) - F_n(k - 1))
                = \lim_{n \to \infty} F_n(k) - \lim_{n \to \infty} F_n(k - 1).
        \end{equation*}
        Rewriting the equation gives us $F(k) - \lim_{n \to \infty} F_n(k) = F(k - 1) - \lim_{n \to \infty} F_n(k - 1)$.
        Recursively applying the equation above gives
        \begin{equation*}
            F(k) - \lim_{n \to \infty} F_n(k) = ...
                = F(-1) - \lim_{n \to \infty} F_n(-1)
                = 0.
        \end{equation*}
        So $F(k) = \lim_{n \to \infty} F_n(k)$.
\end{itemize}


\subsection*{Solution 13}

Let $Z_1, Z_2, ...$ be i.i.d. random variables with probability density function $f$.
Suppose that $P(Z_i > 0) = 1$ and $\lambda = \lim_{x \downarrow 0} f(x) > 0$.
Let $X_n = n \min(Z_1, Z_2, ..., Z_n$.
Note that
\begin{equation*}
    F(x) = \int_{-\infty}^{x} f(t) dt
        = 1 - \int_{x}^{\infty} f(t) dt
        \quad \rightarrow \quad
    F'(0) = \left( 1 - \int_{x}^{\infty} f(t) dt \right)'
        = \lim_{x \downarrow 0} f(x)
        = \lambda > 0.
\end{equation*}
By the Taylor expension of $F$ in $0$ we have
\begin{equation*}
    P(Z_i \geq \frac{x}{n}) = 1 - P(Z_i \leq \frac{x}{n})
        = 1 - F(\frac{x}{n})
        = 1 - (F(0) + \frac{x}{n} F'(0) + O(\frac{x^2}{n^2}))
        = 1 - \frac{x}{n} \lambda + O(\frac{x^2}{n^2}).
\end{equation*}
Such that
\begin{equation*}
    \begin{split}
        F_n(x)
            &= P\left(X_n \leq x\right) \\
            &= P\left(\min(Z_1, Z_2, ..., Z_n) \leq \frac{x}{n}\right) \\
            &= 1 - \prod_{i = 1}^n P\left(Z_i \geq \frac{x}{n}\right) \\
            &= 1 - \prod_{i = 1}^n \left(1 - \frac{x \lambda}{n} + O(\frac{x^2}{n^2})\right) \\
            &= 1 - \left(1 - \frac{x \lambda}{n}\right)^n + O\left(\frac{x^2}{n^2}\right)
            \to 1 - e^{-\lambda x},
    \end{split}
\end{equation*}
as $n \to \infty$.
So $X_n \xrightarrow{D} Z \sim \mathrm{Exp}(\frac{1}{\lambda})$.


\subsection*{Solution 14}

Let $X_1, X_2, ..., X_n \sim \mathrm{Uniform}(0, 1)$.
Define $Y_n = (\overline{X}_n)^2$.
From the weak law of large numbers $\overline{X}_n \xrightarrow{P} \mu = \frac{1}{2}$, hence $\overline{X}_n \xrightarrow{D} \frac{1}{2}$.
Let $g(x) = x^2$, by Theorem 5.5.(g), $Y_n = g(\overline{X}_n) \xrightarrow{D} g(\frac{1}{2}) = \frac{1}{4}$.

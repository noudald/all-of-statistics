\section*{Chapter 2 - Random Variables}

\subsection*{Solution 1}

Let $x^+_1 > x^+_2 > ...$ and $x^-_1 < x^-_2 < ...$ such that $x^+_i \downarrow x$ and $x^-_i \uparrow x$.
Let $A_i = [x^-_i, x^+_i]$ for all $i > 0$.
Note that $A_{i+1} \subset A_i$ for all $i > 0$ and $A_i \to \{x\}$ as $i \to \infty$.
By theorem 1.8 (Continuity of Probability)
$$
P(x) = \lim_{i \to \infty} P(A_i) = \lim_{i \to \infty} (F(x^+_i) - F(x^-_i)) = F(x^+) - F(x^-).
$$

\subsection*{Solution 1 - Alternative}

By lemma 2.15.1 $P(x) = F(x) - F(x^-)$.
By Theorem 2.8.iii $F$ is right continuous, i.e. $F(x) = F(x^+)$, so $P(x) = F(x^+) - F(x^-)$.


\subsection*{Solution 2}

The CDF is given by
$$
F(x) = \left\{ \begin{array}{ll}
    0 & \text{if } x < 2, \\
    \frac{1}{10} & \text{if } 2 \leq x < 3, \\
    \frac{2}{10} & \text{if } 3 \leq x < 5, \\
    1 & \text{if } 5 \leq x
\end{array} \right.
$$
So $P(2 \leq X \leq 4.8) = F(4.8) - F(2^-) = \frac{2}{10}$.


\subsection*{Solution 3}

\begin{enumerate}
\item Let $x^-_1 < x^-_2 < ...$ such that $x^-_i \to x$ and $A_i = (x^-_i, x]$.
Then $A_i \supset A_{i+1}$ for all $i$ and $A_i \to \{x\}$.
By the continuity of the probability function
$$
P(x) = \lim_{i \to \infty} P(A_i) = \lim_{i \to \infty} F(x) - F(x^-_i) = F(x) - F(x^-).
$$
\item $P(x < X \leq y) = P(X \leq y) - P(X \leq x) = F(y) - F(x)$.
\item $P(X > x) = 1 - P(X \leq x) = 1 - F(x)$.
\item Follows directly from continuity of the probability function.
$P(x < X) = P(x \leq X)$ if $P$ is continuous.
\end{enumerate}


\subsection*{Solution 4}

\begin{itemize}
    \item[(a)]
        \begin{equation*}
            F_X(x) = \left\{ \begin{array}{ll}
                0 & \text{if } x \leq 0 \\
                x/4 & \text{if } 0 < x < 1 \\
                1/4 & \text{if } 1 \leq x \leq 3 \\
                1/8(3x - 7) & \text{if } 3 < x < 5 \\
                1 & \text{if } 5 \leq x
            \end{array} \right.
        \end{equation*}
    \item[(b)] For $Y = 1/X$, $F_Y(y) = P(1/X < y) = P(1/y < X) = 1 - P(X < 1/y)$, so
        \begin{equation*}
            F_Y(y) = \left\{ \begin{array}{ll}
                0 & \text{if } y < 1/5 \\
                1 - 1/8(3/y - 7) & \text{if } 1/5 \leq y < 1/3 \\
                3/4 & \text{if } 1/3 \leq y \leq 1 \\
                1 - 1/(4y) & \text{if } 1 < y
            \end{array} \right.
        \end{equation*}
\end{itemize}


\subsection*{Solution 5}

Almost straight from the definition.
As $X$ and $Y$ are discrete random variables:
\begin{itemize}
\item[$\rightarrow$)] $f(x, y) = P(X \in \{x\}, Y \in \{y\}) = P(X \in \{x\})P(Y \in \{y\}) = f(x)f(y)$.
\item[$\leftarrow$)] $P(X \in A, Y \in B) = \sum_{(x, y) \in A \times B} f(x, y) = \sum_{x \in A} \sum_{y \in B} f(x) f(y) = P(X \in A) P(Y \in B)$.
\end{itemize}


\subsection*{Solution 6}

$Y = I_A(X)$ takes the value $1$ if $X \in A$ and $0$ if $X \notin A$.
So $P(Y = 0) = 1 - F_X(A)$ and $P(Y = 1) = F_X(A)$.
This gives the CDF
$$
F_Y(y) = \left\{ \begin{array}{ll}
    0 & \text{if } y < 0 \\
    1 - F_X(A) & \text{if } 0 \leq y < 1 \\
    1 & \text{if } 1 \leq y
\end{array} \right.
$$


\subsection*{Solution 7}

Let $Z = \mathrm{min}(X, Y)$, $X, Y \sim \mathrm{Uniform}(0, 1)$, then
\begin{equation*}
\begin{split}
F_Z(z) &= P(Z < z) \\
    &= 1 - P(Z > z) \\
    &= 1 - P(X > z \text{ and } Y > z) \\
    &= 1 - P(X > z)P(Y > z) \\
    &= 1 -(1 - F_X(z))(1 - F_Y(z))
    = 1 - (1 - z)^2.
\end{split}
\end{equation*}
Taking the derivative gives the probability distribution function
$$
f_Z(z) = 2 (1 - z).
$$


\subsection*{Solution 8}

Let $X$ be a random variable with CDF $F$.
Define $X^+ = \mathrm{max}(0, X)$.
Note that $P(X^+ < 0) = 0$ and $P(X^+ = 0) = P(X \leq 0) = F(0)$.
Therefore
\begin{equation*}
    F_+(x) = \left\{ \begin{array}{ll}
        0 & \text{if } x < 0 \\
        F(x) & \text{if } x \geq 0
    \end{array} \right.
\end{equation*}


\subsection*{Solution 9}

Let $X \sim \mathrm{Exp}(\beta)$.
We have $f(x) = \beta^{-1} \exp(-\beta^{-1}x)$.
Then
$$
F_X(x) = \int_{-\infty}^x f(t)dt
    = \int_{-\infty}^x \frac{1}{\beta} \exp\left(-\frac{t}{\beta}\right) dt
    = 1 - \exp\left(-\frac{x}{\beta}\right).
$$
Solving $x$ for $q = F_X(x) = 1 - \exp(-\beta^{-1}x)$ yields $x = F_X^{-1}(q) = -\beta \log(1 - q)$.


\subsection*{Solution 10}

Follows almost from the definition.
\begin{equation*}
\begin{split}
P(g(X) \in A, h(Y) \in B)
    &= P(X \in g^{-1}(A), Y \in h^{-1}(B)) \\
    &= P(X \in g^{-1}(A)) P(Y \in h^{-1}(B))
    = P(g(X) \in A) P(h(Y) \in B).
\end{split}
\end{equation*}


\subsection*{Solution 11}

(a) $P(X = 1, Y = 0) = p$, but $P(X = 1) = p$ and $P(Y = 0) = p$, so $P(X = 1)P(Y = 1) = p^2 \neq p$ (assuming $p > 0$).
Therefore $X$ and $Y$ are dependent.

(b) Let $N \sim \mathrm{Poisson}(\lambda)$, flip $N$ coins and let $X$ be the number of heads, $Y$ the number of tails.
We have
\begin{equation*}
\begin{split}
P(X = i, Y = j)
    &= f_{\lambda}(N) \binom{N}{i} p^i (1 - p)^{N - i} \\
    &= f_{\lambda}(i + j) \binom{i + j}{i} p^i (1 - p)^j \\
    &= e^{-\lambda} \frac{\lambda^{i + j}}{(i + j)!} \frac{(i + j)!}{i! j!} p^i (1 - p)^j \\
    &= e^{-\lambda} \frac{\lambda^i p^i}{i!} \frac{\lambda^j (1 - p)^j}{j!} = g(i) h(j),
\end{split}
\end{equation*}
where $g(i) = e^{-\lambda} \lambda^i p^i / i!$ and $h(j) = \lambda^j (1 - p)^j / j!$.
By Theorem 2.33 $X$ and $Y$ are independent.

Here follows a remark, as the results of this exercise is counter intuitive (for me).
Suppose you toss a coin $50$ times.
Let $X$ be the number of heads and $Y$ the number of tails.
If $X = 25$, what is $Y$? Easy, $Y = 50 - X = 25$.
If $X = 4$, $Y = 50 - 4 = 46$.
Or $X = 48$, then $Y = 50 - 48 = 2$.
$X$ and $Y$ are entirely correlated.
If I know $X$, I also know $Y$.

Now let me toss a coin every time I receive an email in my mailbox.
I count the number of heads and tails during the day.
Let $X$ be the number of heads and $Y$ the number of tails of that day.
I receive around $50$ mails a day, and the exact number of mails per day follows a Poisson distribution with $\lambda = 50$.
After one day I count $X = 25$ heads, how many tails do I expect to have counted?
Intuitively I expect $Y = 25$ tails, which is correct.
But suppose now that $X = 4$?
Because $X$ is a low number, I expect that I have not received many mails that day, and therefore flipped few coins, so $E(E(Y | X = 4)) = 4$ would be an intuitive guess.
Similar, if $X = 48$, I probably received many mails, flipped a lot of coins, so $E(E(Y | X = 48)) = 48$ seems reasonable.

But this is not the case.
As we saw in the exercise, $X$ and $Y$ are independent.
Therefore
$$
E(Y | X = x) = \int y f(y|x) dy = \int y f(y) dy = E(Y),
$$
which shows that $E(E(Y | X = x)) = E(Y) = \frac{\lambda}{2} = 25$.
In particular, $E(E(Y | X = 25)) = E(E(Y | X = 4)) = E(E(Y | X = 48)) = 25$.
The number of heads will give you absolutely no information about the number of tails that occured during the day.
This is an extreme case where a small change, going from $N = 50$ to $N \sim \mathrm{Poisson}(50)$, turns entirely dependent random variables into completely independent random variables.


\subsection*{Solution 12}

Let $f(x, y) = g(x) h(y)$ for all $x, y$.
Note
$$
1 = \lim_{x, y \to infty} F(x, y) = \int f(x, y) dx dy = \int g(x) dx \int h(y) dy = A_g A_h.
$$
Now
\begin{equation*}
    \begin{split}
        f_X(x) &= \int f(x, y) dy = \int g(x) h(y) dy = A_h g(x), \\
        f_Y(y) &= \int f(x, y) dx = \int g(x) h(y) dx = A_g h(y).
    \end{split}
\end{equation*}
We have $f(x, y) = g(x) h(y) = A_g A_h g(x) h(y) = (A_h g(x))(A_g h(y)) = f_X(x) f_Y(y)$, which shows that $X$ and $Y$ are independent.


\subsection*{Solution 13}

Let $X \sim \mathrm{Normal}(0, 1)$ and $Y = \exp(X)$.

(a) We first calculate $F_Y$,
\begin{equation*}
F_Y(y) = P(Y < y)
    = P(X < \log(y))
    = \Phi(\log(y))
    = \frac{1}{\sqrt{2\pi}} \int_{-\infty}^{\log(y)} \exp\left(-\frac{1}{2} t^2\right) dt
\end{equation*}
By the fundamental theorem of calculus
$$
f_Y(y) = \frac{1}{\sqrt{2\pi}} \exp\left(-\frac{1}{2} \log(y)^2\right).
$$

(b) See code.


\subsection*{Solution 14}

Let $A_r = \{(x, y) : x^2 + y^2 \leq r^2\}$ be the circle with radius $r$.
We have $F_R(r) = \mathrm{Opp}(A_r) / \mathrm{Opp}(A_1) = r^2 \pi / \pi = r^2$.
So $f_R(r) = 2r$.


\subsection*{Solution 15}

Let $X \sim F$ and $F$ continuous, strictly increasing, i.e., $x_1 < x_2$, then $F(x_1) < F(x_2)$.
Let $Y = F(X)$.

(a) $F_Y(y) = P(Y \leq y) = P(F(X) \leq y) = P(X \leq F^{-1}(y)) = F(F^{-1}(y)) = y$.
So $Y \sim \mathrm{Uniform}(0, 1)$.

(b) Let $U \sim \mathrm{Uniform}(0, 1)$ and $X = F^{-1}(U)$.
Then $F_X(x) = P(X \leq x) = P(F^{-1}(U) \leq x) = P(U \leq F(x)) = F(x)$.
So $X \sim F$.

(c) Take $F(x) = 1 - \exp(-\frac{x}{\beta}) = u$, then $F^{-1}(u) = -\beta \log(1 - u)$.
So if $U \sim \mathrm{Uniform}(0, 1)$, then $X = F^{-1}(U) = -\beta \log(1 - U) \sim \mathrm{Exp}(\beta)$.
See code for an implementation.


\subsection*{Solution 16}

$X \sim \mathrm{Poisson}(\lambda)$, $Y \sim \mathrm{Poisson}(\mu)$.
Let $n = X + Y$.
\begin{equation*}
    P(X = x | X + Y = n) \propto P(X = x, X + Y = n)
        = e^{-\lambda} \frac{\lambda^x}{x!} e^{-\mu} \frac{\mu^{n-x}}{(n - x)!}
        \propto \frac{\lambda^x \mu^{n-x}}{x!(n-x!)}
        \propto \binom{n}{x} \theta^x (1 - \theta)^x,
\end{equation*}
where
\begin{equation*}
    \theta = \frac{\lambda}{\lambda + \mu}.
\end{equation*}
So $X|X+Y \sim \mathrm{Binomial}(n, \theta)$.


\subsection*{Solution 17}

We have
\begin{equation*}
    f_Y\left(\frac{1}{2}\right) = \int_0^1 f\left(x, \frac{1}{2}\right) dx = \int_0^1 c\left(x + \frac{1}{4}\right) dx = \frac{3}{4}c.
\end{equation*}
Using the definition of the conditional probability density function
\begin{equation*}
    P\left(X < \frac{1}{2} \left| Y = \frac{1}{2} \right.\right)
        = \int_0^{\frac{1}{2}} f(x | y = 1/2) dx
        = \int_0^{\frac{1}{2}} \frac{f(x, 1/2)}{f(1/2)} dx
        = \frac{\frac{1}{4}c}{\frac{3}{4}c}
        = \frac{1}{3}.
\end{equation*}
